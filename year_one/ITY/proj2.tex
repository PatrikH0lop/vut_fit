%document
\documentclass[11pt,twocolumn,a4paper]{article}
\usepackage[left=1.5cm,text={18cm, 25cm},top=2.5cm]{geometry}
\usepackage[czech]{babel}
\usepackage{times}
\usepackage[utf8]{inputenc}
\usepackage[T1]{fontenc}
%matematicke znacenie
\usepackage{amsmath}
\usepackage{amsthm}
\usepackage{amsfonts}

%Vysadenie definicii 
\theoremstyle{definition}
\newtheorem{mydef}{Definice}[section]
\theoremstyle{plain}
\newtheorem{myalg}[mydef]{Algoritmus}
\newtheorem{mylem}{Věta}

%Zaciatok dokumentu
\begin{document}
%Uvodna strana
\begin{titlepage}
 \begin{center}
 {
  \Huge
  \textsc{Fakulta informačních technologií\\
  \vspace{-3.5mm}
  Vysoké učení technické v~Brně}\\
 }
  \vspace{\stretch{0.382}}
 {
  \LARGE
  Typografie a~publikování\,--\,2. projekt\\
  \vspace{-2.05mm}
  Sazba dokumentů a~matematických výrazů\\
 }
  \vspace{\stretch{0.618}}
 \end{center}
{\Large 2017 \hfill Patrik Holop}
\end{titlepage}

\section*{Úvod}
V~této úloze si vyzkoušíme sazbu titulní strany, matematických vzorců, prostředí a~dalších textových struktur obvyklých pro technicky zaměřené texty (například rovnice \eqref{eq:1} nebo definice \ref{def1} na straně \pageref{def1}.

Na titulní straně je využito sázení nadpisu podle optického středu s~využitím zlatého řezu. Tento postup byl probírán na přednášce.


\section{Matematický text}

Nejprve se podíváme na sázení matematických symbolů a~výrazů v~plynulém textu. Pro množinu $V$ označuje $\mbox{card}(V)$ kardinalitu $V$.
Pro množinu $V$ reprezentuje $V^*$ volný monoid generovaný množinou $V$ s~operací konkatenace.
Prvek identity ve volném monoidu $V^*$ značíme symbolem $\varepsilon$.
Nechť $V^+=V^*-\{\varepsilon\}$. Algebraicky je tedy $V^+$ volná pologrupa generovaná množinou $V$ s~operací konkatenace.
Konečnou neprázdnou množinu $V$ nazvěme \textit{abeceda}.
Pro $w\in V^*$ označuje $|w|$ délku řetězce $w$. Pro $W\subseteq V$ označuje $\mbox{occur}(w,W)$ počet výskytů symbolů z~$W$ v~řetězci $w$ a~$\mbox{sym}(w, i)$ určuje $i$-tý symbol řetězce $w$; například $\mbox{sym}(abcd,3) = c$.

Nyní zkusíme sazbu definic a vět s~využitím balíku \texttt{amsthm}.

\theoremstyle{definition} 
\begin{mydef} \label{def1} 
{\textit{Bezkontextová gramatika}} je čtveřice $G = (V,T,P,S)$, kde $V$ je totální abeceda,
$T\subseteq V$ je abeceda terminálů, $S\in (V-T)$ je startující symbol a~$P$ je konečná množina \textit{pravidel} 
tvaru $q\colon A\rightarrow \alpha$, kde $A\in (V-T)$, $\alpha\in V^*$ a~$q$ je návěští tohoto pravi\-dla. Nechť $N=V-T$ značí abecedu neterminálů.
Po\-kud $q\colon A\rightarrow \alpha\in P$, $\gamma,\delta\in V^*$, $G$ provádí derivační krok z~$\gamma A\delta$ do $\gamma\alpha\delta$ podle pravidla $q\colon A\rightarrow A$, sym\-bolicky píšeme 
$\gamma A\delta \Rightarrow \gamma\alpha\delta \ [q\colon A\rightarrow\alpha]$ nebo zjed\-nodušeně $\gamma A\delta \Rightarrow \gamma\alpha\delta$. Standardním způsobem definujeme $\Rightarrow^m$, kde $m\geq0$. Dále definu\-jeme tranzitivní uzávěr $\Rightarrow^+$ a~tranzitivně-reflexivní uzávěr $\Rightarrow^*$.
\end{mydef}

Algoritmus můžeme uvádět podobně jako definice textově, nebo využít pseudokódu vysázeného ve vhodném prostředí (například \texttt{algorithm2e}).

\theoremstyle{plain}
\begin{myalg}
Algoritmus pro ověření bezkontextovosti gramatiky. Mějme gramatiku $G = (N, T, P, S)$.
\begin{enumerate}
 \item Pro každé pravidlo $p\in P$ proveď test, zda $p$ na levé straně obsahuje právě jeden symbol z~$N$.\label{step1}
 \item Pokud všechna pravidla splňují podmínku z~kroku \ref{step1}, tak je gramatika $G$ bezkontextová.
\end{enumerate}
\end{myalg}

\theoremstyle{definition}
\begin{mydef}
\textit{Jazyk} definovaný gramatikou $G$ definujeme jako $L(G) = \{w\in T^*|S\Rightarrow^* w\}$.
\end{mydef}

\subsection{Podsekce obsahující větu}

\theoremstyle{definition}
\begin{mydef}
Nechť $L$ je libovolný jazyk. $L$ je \textit{bezkontextový jazyk}, když a~jen když $L = L(G)$, kde $G$ je libovolná bezkontextová gramatika.
\end{mydef}

\theoremstyle{definition}
\begin{mydef}
Množinu $\mathcal{L}_{CF} = \{L|L$ je bezkontextový jazyk\} nazýváme \textit{třídou bezkontextových jazyků}.
\end{mydef}

\theoremstyle{plain}
\begin{mylem}\label{lem:1}
Nechť $L_{abc}=\{a^nb^nc^n|n\geq0\}$. Platí, že $L_{abc}\notin \mathcal{L}_{CF}$.
\end{mylem}

\begin{proof}
Důkaz se provede pomocí Pumping lemma pro bezkontextové jazyky, kdy ukážeme, že není možné, aby platilo, což bude implikovat pravdivost věty \ref{lem:1}.
\end{proof}

\section{Rovnice a odkazy}

Složitější matematické formulace sázíme mimo plynulý text. Lze umístit několik výrazů na jeden řádek, ale pak je třeba tyto vhodně oddělit, například příkazem \verb|\quad|. 

\begin{equation*}
\sqrt[x^2]{y^{3}_{0}}\quad \mathbb{N}=\{0,1,2, \dots \}\quad x^{y^y}\neq x^{yy}\quad z_{i_j}\not\equiv  z_{ij} 
\end{equation*}

V~rovnici \eqref{eq:1} jsou využity tři typy závorek s~různou explicitně definovanou velikostí.

\begin{eqnarray}
\bigg\{ \Big[ \big(a+b\big)*c\Big]^d+1\bigg\} & = & x \label{eq:1}\\
\lim\limits_{x\to\infty} \frac{\sin^2x + \cos^2x}{4} & = & y \nonumber
\end{eqnarray}

V~této větě vidíme, jak vypadá implicitní vysázení limity $\lim_{n\to\infty} f(n)$ v~normálním odstavci textu. Podobně je to i s~dalšími symboly jako $\sum_{1}^{n}$ či $\bigcup_{A\in \mathcal{B}}$ . V~případě vzorce $\lim\limits_{x\to 0} \frac{\sin x}{x} = 1$ jsme si vynutili méně úspornou sazbu příkazem \verb|\limits|.

\begin{eqnarray}
\int\limits_{a}^{b} f(x) \mathrm{d}x & = & -\int_{a}^{b} f(x) \mathrm{d}x\\ 
\Big(\sqrt[5]{x^4}\Big)' = \Big(x^\frac{4}{5}\Big)' & = & \frac{4}{5}x^{\frac{1}{5}} = \frac{4}{5\sqrt[5]{x}}\\ 
\overline{\overline{A\vee B}} & = & \overline{\overline{A} \wedge \overline{B}} 
\end{eqnarray}


\section{Matice}

Pro sázení matic se velmi často používá prostředí \texttt{array} a~závorky (\verb|\left|,\verb|\right|). 

\[ \left( \begin{array}{cc}
a+b & b-a\\
\widehat{\xi+\omega} & \hat{\pi}\\
\vec{a} & \overleftrightarrow{AC}\\
0 & \beta
\end{array} \right) \]

\[ \textbf{A} =  \left|\left| \begin{array}{cccc}
a_{11} & a_{12} & \cdots & a_{1n}\\
a_{21} & a_{22} & \cdots & a_{2n}\\
\vdots & \vdots & \ddots & \vdots\\
a_{m1} & a_{m2} & \cdots & a_{mn}\\
\end{array} \right|\right| \]

\[ \left| \begin{array}{cc}
t & u\\
v~& w
\end{array} \right| = tw - uv \]

Prostředí \texttt{array} lze úspěšně využít i~jinde.

$$ \binom{n}{k} = \left\{
\begin{array}{l l}
\frac{n!}{k!(n-k)!} & \text{pro } 0\leq k\leq n \\
0 & \text{pro } k\leq 0 \text{ nebo } k\,>\,n
\end{array} \right. $$

\section{Závěrem}

V~případě, že budete potřebovat vyjádřit matematickou konstrukci nebo symbol a~nebude se Vám dařit jej nalézt v~samotném \LaTeX u, doporučuji prostudovat možnosti ba\-líku maker \AmS-\LaTeX .
Analogická poučka platí obec\-ně pro jakoukoli konstrukci v~\TeX u.

\end{document}